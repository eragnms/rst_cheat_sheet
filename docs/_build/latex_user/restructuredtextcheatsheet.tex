%% Generated by Sphinx.
\def\sphinxdocclass{report}
\documentclass[letterpaper,10pt,english]{sphinxmanual}
\ifdefined\pdfpxdimen
   \let\sphinxpxdimen\pdfpxdimen\else\newdimen\sphinxpxdimen
\fi \sphinxpxdimen=.75bp\relax
\ifdefined\pdfimageresolution
    \pdfimageresolution= \numexpr \dimexpr1in\relax/\sphinxpxdimen\relax
\fi
%% let collapsible pdf bookmarks panel have high depth per default
\PassOptionsToPackage{bookmarksdepth=5}{hyperref}
%% turn off hyperref patch of \index as sphinx.xdy xindy module takes care of
%% suitable \hyperpage mark-up, working around hyperref-xindy incompatibility
\PassOptionsToPackage{hyperindex=false}{hyperref}
%% memoir class requires extra handling
\makeatletter\@ifclassloaded{memoir}
{\ifdefined\memhyperindexfalse\memhyperindexfalse\fi}{}\makeatother

\PassOptionsToPackage{booktabs}{sphinx}
\PassOptionsToPackage{colorrows}{sphinx}

\PassOptionsToPackage{warn}{textcomp}

\catcode`^^^^00a0\active\protected\def^^^^00a0{\leavevmode\nobreak\ }
\usepackage{cmap}
\usepackage{fontspec}
\defaultfontfeatures[\rmfamily,\sffamily,\ttfamily]{}
\usepackage{amsmath,amssymb,amstext}
\usepackage{polyglossia}
\setmainlanguage{english}



    \setmainfont{Libertinus Serif}        % full OpenType with math companion
    \setsansfont{Libertinus Sans}
    \setmonofont{Fira Mono}
    \usepackage{unicode-math}
    \setmathfont{Libertinus Math}

    \usepackage{fontawesome5}
    % \faExclamationTriangle → ⚠, etc. (still not really text to copy though)
    


\usepackage[Bjarne]{fncychap}
\usepackage{sphinx}

\fvset{fontsize=auto}
\usepackage{geometry}


% Include hyperref last.
\usepackage{hyperref}
% Fix anchor placement for figures with captions.
\usepackage{hypcap}% it must be loaded after hyperref.
% Set up styles of URL: it should be placed after hyperref.
\urlstyle{same}

\addto\captionsenglish{\renewcommand{\contentsname}{Guides}}

\usepackage{sphinxmessages}
\setcounter{tocdepth}{0}

\XeTeXgenerateactualtext=1

\title{reStructuredText Cheat Sheet}
\date{Nov 11, 2025}
\release{}
\author{Mats Gustafsson}
\newcommand{\sphinxlogo}{\vbox{}}
\renewcommand{\releasename}{}
\makeindex
\begin{document}

\pagestyle{empty}
\sphinxmaketitle
\pagestyle{plain}
\sphinxtableofcontents
\pagestyle{normal}
\phantomsection\label{\detokenize{index::doc}}


\sphinxstepscope


\chapter{reStructuredText cheat sheet}
\label{\detokenize{rst_cheat_sheet:restructuredtext-cheat-sheet}}\label{\detokenize{rst_cheat_sheet::doc}}
\index{reStructuredText@\spxentry{reStructuredText}!format@\spxentry{format}}\index{Sphinx@\spxentry{Sphinx}!cross\sphinxhyphen{}references@\spxentry{cross\sphinxhyphen{}references}}\index{tables@\spxentry{tables}}\index{headings@\spxentry{headings}}\index{index entries@\spxentry{index entries}}\ignorespaces 
\sphinxAtStartPar
This section summarizes commonly used reStructuredText (RST) and Sphinx syntax for
editing and maintaining documentation.


\section{Common Conventions}
\label{\detokenize{rst_cheat_sheet:common-conventions}}\begin{itemize}
\item {} 
\sphinxAtStartPar
Use \sphinxcode{\sphinxupquote{code\sphinxhyphen{}block:: bash}} for shell commands.

\item {} 
\sphinxAtStartPar
Use \sphinxcode{\sphinxupquote{code\sphinxhyphen{}block:: python}} for code examples.

\item {} 
\sphinxAtStartPar
Inline code or filenames: \sphinxcode{\sphinxupquote{example.py}}

\item {} 
\sphinxAtStartPar
Bold and italics: \sphinxstylestrong{bold}, \sphinxstyleemphasis{italic}

\item {} 
\sphinxAtStartPar
Lists:
\sphinxhyphen{} Unordered: \sphinxcode{\sphinxupquote{\sphinxhyphen{}}} or \sphinxcode{\sphinxupquote{*}}
\sphinxhyphen{} Ordered: \sphinxcode{\sphinxupquote{1.}} \sphinxcode{\sphinxupquote{2.}} etc.

\item {} 
\sphinxAtStartPar
Avoid overusing raw HTML; Sphinx supports nearly all formatting needs natively.

\end{itemize}

\sphinxAtStartPar
\sphinxstylestrong{Rule of thumb:} aim for 2–5 index entries per RST file, more only for
very technical chapters (like \sphinxcode{\sphinxupquote{usage.rst}}).


\section{Headings}
\label{\detokenize{rst_cheat_sheet:headings}}
\sphinxAtStartPar
Our heading hierarchy follows this exact order:
\begin{itemize}
\item {} 
\sphinxAtStartPar
Top Level (e.g. document title) =

\item {} 
\sphinxAtStartPar
Second Level \sphinxhyphen{}

\item {} 
\sphinxAtStartPar
Third Level +

\item {} 
\sphinxAtStartPar
Fourth Level \#

\item {} 
\sphinxAtStartPar
Fifth Level “

\end{itemize}

\sphinxAtStartPar
Always keep the underline \sphinxstylestrong{at least as long as} the heading text.


\section{Cross\sphinxhyphen{}references}
\label{\detokenize{rst_cheat_sheet:cross-references}}
\index{references@\spxentry{references}!internal links@\spxentry{internal links}}\index{cross\sphinxhyphen{}references@\spxentry{cross\sphinxhyphen{}references}!syntax@\spxentry{syntax}}\ignorespaces 
\sphinxAtStartPar
References connect sections or terms so they can be linked from elsewhere.

\sphinxAtStartPar
\sphinxstylestrong{Internal section reference}
\begin{enumerate}
\sphinxsetlistlabels{\arabic}{enumi}{enumii}{}{.}%
\item {} 
\sphinxAtStartPar
Define a label before the heading:

\begin{sphinxVerbatim}[commandchars=\\\{\}]
\PYG{p}{..} \PYG{n+nt}{\PYGZus{}my\PYGZhy{}section\PYGZhy{}label:}

\PYG{g+gh}{My Section Title}
\PYG{g+gh}{\PYGZhy{}\PYGZhy{}\PYGZhy{}\PYGZhy{}\PYGZhy{}\PYGZhy{}\PYGZhy{}\PYGZhy{}\PYGZhy{}\PYGZhy{}\PYGZhy{}\PYGZhy{}\PYGZhy{}\PYGZhy{}\PYGZhy{}\PYGZhy{}}
\end{sphinxVerbatim}

\item {} 
\sphinxAtStartPar
Refer to it from anywhere using:

\begin{sphinxVerbatim}[commandchars=\\\{\}]
See \PYG{n+na}{:ref:}\PYG{n+nv}{`my\PYGZhy{}section\PYGZhy{}label`} for details.
\end{sphinxVerbatim}

\end{enumerate}

\sphinxAtStartPar
\sphinxstylestrong{File reference}
\begin{quote}

\begin{sphinxVerbatim}[commandchars=\\\{\}]
\PYG{n+na}{:doc:}\PYG{n+nv}{`../path/to/other\PYGZus{}file`}
\end{sphinxVerbatim}
\end{quote}

\sphinxAtStartPar
\sphinxstylestrong{Inline hyperlink}
\begin{quote}

\begin{sphinxVerbatim}[commandchars=\\\{\}]
The \PYG{l+s}{`Sphinx documentation }\PYG{l+s+si}{\PYGZlt{}https://www.sphinx\PYGZhy{}doc.org/\PYGZgt{}}\PYG{l+s}{`\PYGZus{}} is excellent.
\end{sphinxVerbatim}
\end{quote}


\section{Index entries}
\label{\detokenize{rst_cheat_sheet:index-entries}}
\index{indexing@\spxentry{indexing}!syntax@\spxentry{syntax}}\index{keywords@\spxentry{keywords}!documentation@\spxentry{documentation}}\ignorespaces 
\sphinxAtStartPar
Use \sphinxcode{\sphinxupquote{.. index::}} to add searchable keywords.

\sphinxAtStartPar
\sphinxstylestrong{Examples:}

\begin{sphinxVerbatim}[commandchars=\\\{\}]
\PYG{p}{..} \PYG{o+ow}{index}\PYG{p}{::}
   single: Docker; build container
   pair: direction of arrival; DoA
   triple: machine learning; classification; examples
\end{sphinxVerbatim}

\sphinxAtStartPar
Place index entries \sphinxstylestrong{immediately below} the relevant section title for accurate linking.

\sphinxAtStartPar
\sphinxstylestrong{Index only what’s useful}
\begin{itemize}
\item {} 
\sphinxAtStartPar
Do not index every heading or parameter.

\item {} 
\sphinxAtStartPar
Index \sphinxstylestrong{concepts people might look up later} such as tools, algorithms,
file formats, configuration options, and important scripts.

\end{itemize}

\sphinxAtStartPar
\sphinxstylestrong{Use specific keywords}
\begin{itemize}
\item {} 
\sphinxAtStartPar
Prefer concrete terms: \sphinxcode{\sphinxupquote{CFAR\sphinxhyphen{}DoA}} or \sphinxcode{\sphinxupquote{Hailo\sphinxhyphen{}8L}} instead of \sphinxcode{\sphinxupquote{code}} or
\sphinxcode{\sphinxupquote{setup}}.

\end{itemize}

\sphinxAtStartPar
\sphinxstylestrong{Group related terms}
\begin{itemize}
\item {} 
\sphinxAtStartPar
Use \sphinxcode{\sphinxupquote{;}} to build hierarchy:

\begin{sphinxVerbatim}[commandchars=\\\{\}]
\PYG{p}{..} \PYG{o+ow}{index}\PYG{p}{::}
   single: audio; conversion
   single: audio; resampling
\end{sphinxVerbatim}

\end{itemize}

\sphinxAtStartPar
\sphinxstylestrong{Use pairs/triples for synonyms}
\begin{itemize}
\item {} 
\sphinxAtStartPar
If a concept has multiple names, use \sphinxcode{\sphinxupquote{pair}} or \sphinxcode{\sphinxupquote{triple}} so it appears in
all relevant places in the index:

\begin{sphinxVerbatim}[commandchars=\\\{\}]
\PYG{p}{..} \PYG{o+ow}{index}\PYG{p}{::}
   pair: direction of arrival; DoA
\end{sphinxVerbatim}

\end{itemize}

\sphinxAtStartPar
\sphinxstylestrong{Index tools, not every option}
\begin{itemize}
\item {} 
\sphinxAtStartPar
For commands or scripts, add one index entry for the tool.

\item {} 
\sphinxAtStartPar
Avoid indexing every CLI flag—search handles that.

\end{itemize}


\section{Tables}
\label{\detokenize{rst_cheat_sheet:tables}}
\index{tables@\spxentry{tables}!syntax@\spxentry{syntax}}\index{grid tables@\spxentry{grid tables}}\ignorespaces 
\sphinxAtStartPar
Use \sphinxstylestrong{grid tables} for clarity and column alignment.

\sphinxAtStartPar
\sphinxstylestrong{Example:}

\begin{sphinxVerbatim}[commandchars=\\\{\}]
+\PYGZhy{}\PYGZhy{}\PYGZhy{}\PYGZhy{}\PYGZhy{}\PYGZhy{}\PYGZhy{}\PYGZhy{}\PYGZhy{}\PYGZhy{}\PYGZhy{}\PYGZhy{}\PYGZhy{}\PYGZhy{}\PYGZhy{}\PYGZhy{}+\PYGZhy{}\PYGZhy{}\PYGZhy{}\PYGZhy{}\PYGZhy{}\PYGZhy{}\PYGZhy{}\PYGZhy{}\PYGZhy{}\PYGZhy{}\PYGZhy{}\PYGZhy{}\PYGZhy{}\PYGZhy{}\PYGZhy{}\PYGZhy{}+\PYGZhy{}\PYGZhy{}\PYGZhy{}\PYGZhy{}\PYGZhy{}\PYGZhy{}\PYGZhy{}\PYGZhy{}\PYGZhy{}\PYGZhy{}\PYGZhy{}\PYGZhy{}\PYGZhy{}\PYGZhy{}\PYGZhy{}\PYGZhy{}+
\PYG{o}{|} Column A       | Column B       | Column C       |
+================+================+================+
\PYG{o}{|} Item 1         | Description 1  | Value 1        |
+\PYGZhy{}\PYGZhy{}\PYGZhy{}\PYGZhy{}\PYGZhy{}\PYGZhy{}\PYGZhy{}\PYGZhy{}\PYGZhy{}\PYGZhy{}\PYGZhy{}\PYGZhy{}\PYGZhy{}\PYGZhy{}\PYGZhy{}\PYGZhy{}+\PYGZhy{}\PYGZhy{}\PYGZhy{}\PYGZhy{}\PYGZhy{}\PYGZhy{}\PYGZhy{}\PYGZhy{}\PYGZhy{}\PYGZhy{}\PYGZhy{}\PYGZhy{}\PYGZhy{}\PYGZhy{}\PYGZhy{}\PYGZhy{}+\PYGZhy{}\PYGZhy{}\PYGZhy{}\PYGZhy{}\PYGZhy{}\PYGZhy{}\PYGZhy{}\PYGZhy{}\PYGZhy{}\PYGZhy{}\PYGZhy{}\PYGZhy{}\PYGZhy{}\PYGZhy{}\PYGZhy{}\PYGZhy{}+
\PYG{o}{|} Item 2         | Description 2  | Value 2        |
+\PYGZhy{}\PYGZhy{}\PYGZhy{}\PYGZhy{}\PYGZhy{}\PYGZhy{}\PYGZhy{}\PYGZhy{}\PYGZhy{}\PYGZhy{}\PYGZhy{}\PYGZhy{}\PYGZhy{}\PYGZhy{}\PYGZhy{}\PYGZhy{}+\PYGZhy{}\PYGZhy{}\PYGZhy{}\PYGZhy{}\PYGZhy{}\PYGZhy{}\PYGZhy{}\PYGZhy{}\PYGZhy{}\PYGZhy{}\PYGZhy{}\PYGZhy{}\PYGZhy{}\PYGZhy{}\PYGZhy{}\PYGZhy{}+\PYGZhy{}\PYGZhy{}\PYGZhy{}\PYGZhy{}\PYGZhy{}\PYGZhy{}\PYGZhy{}\PYGZhy{}\PYGZhy{}\PYGZhy{}\PYGZhy{}\PYGZhy{}\PYGZhy{}\PYGZhy{}\PYGZhy{}\PYGZhy{}+
\end{sphinxVerbatim}

\sphinxAtStartPar
\sphinxstylestrong{Tips:}
\begin{itemize}
\item {} 
\sphinxAtStartPar
Align \sphinxcode{\sphinxupquote{+}} and \sphinxcode{\sphinxupquote{|}} vertically for clean rendering.

\item {} 
\sphinxAtStartPar
Use \sphinxcode{\sphinxupquote{=}} in the header separator row.

\item {} 
\sphinxAtStartPar
Use monospace font (\sphinxcode{\sphinxupquote{code\sphinxhyphen{}block}}) when showing RST examples.

\end{itemize}


\section{Editing Box\sphinxhyphen{}Drawing Diagrams in Emacs}
\label{\detokenize{rst_cheat_sheet:editing-box-drawing-diagrams-in-emacs}}
\index{Emacs@\spxentry{Emacs}!box\sphinxhyphen{}drawing@\spxentry{box\sphinxhyphen{}drawing}}\index{Doom Emacs@\spxentry{Doom Emacs}!Unicode@\spxentry{Unicode}}\index{reStructuredText@\spxentry{reStructuredText}!diagrams editing@\spxentry{diagrams editing}}\ignorespaces 
\sphinxAtStartPar
When editing Unicode box\sphinxhyphen{}drawing diagrams in \sphinxstylestrong{Emacs} or \sphinxstylestrong{Doom Emacs}, make sure
your buffer is in UTF\sphinxhyphen{}8 encoding and that you use a monospace font for alignment.


\subsection{Check or enforce UTF\sphinxhyphen{}8}
\label{\detokenize{rst_cheat_sheet:check-or-enforce-utf-8}}
\sphinxAtStartPar
Most modern Emacs configurations (including Doom) default to UTF\sphinxhyphen{}8, but you can
verify or set it manually:

\begin{sphinxVerbatim}[commandchars=\\\{\}]
\PYG{p}{(}\PYG{k}{setq\PYGZhy{}default}\PYG{+w}{ }\PYG{n+nv}{buffer\PYGZhy{}file\PYGZhy{}coding\PYGZhy{}system}\PYG{+w}{ }\PYG{l+s+ss}{\PYGZsq{}utf\PYGZhy{}8\PYGZhy{}unix}\PYG{p}{)}
\end{sphinxVerbatim}

\sphinxAtStartPar
If a file opens in another encoding, fix it interactively:

\begin{sphinxVerbatim}[commandchars=\\\{\}]
C\PYGZhy{}x RET f utf\PYGZhy{}8 RET
\end{sphinxVerbatim}


\subsection{Insert box\sphinxhyphen{}drawing characters}
\label{\detokenize{rst_cheat_sheet:insert-box-drawing-characters}}
\sphinxAtStartPar
You can insert any Unicode character by name or code point:

\begin{sphinxVerbatim}[commandchars=\\\{\}]
C\PYGZhy{}x 8 RET BOX DRAWINGS LIGHT HORIZONTAL RET
C\PYGZhy{}x 8 RET 2500 RET
\end{sphinxVerbatim}

\sphinxAtStartPar
This inserts \sphinxcode{\sphinxupquote{─}}.


\subsection{Common characters}
\label{\detokenize{rst_cheat_sheet:common-characters}}

\begin{savenotes}\sphinxattablestart
\sphinxthistablewithglobalstyle
\centering
\begin{tabulary}{\linewidth}[t]{TTT}
\sphinxtoprule
\sphinxstyletheadfamily 
\sphinxAtStartPar
Symbol
&\sphinxstyletheadfamily 
\sphinxAtStartPar
Code
&\sphinxstyletheadfamily 
\sphinxAtStartPar
Insert command
\\
\sphinxmidrule
\sphinxtableatstartofbodyhook
\sphinxAtStartPar
\sphinxcode{\sphinxupquote{─}}
\sphinxcode{\sphinxupquote{│}}
\sphinxcode{\sphinxupquote{┌}}
\sphinxcode{\sphinxupquote{┐}}
\sphinxcode{\sphinxupquote{└}}
\sphinxcode{\sphinxupquote{┘}}
\sphinxcode{\sphinxupquote{┬}}
\sphinxcode{\sphinxupquote{┴}}
\sphinxcode{\sphinxupquote{┼}}
&
\sphinxAtStartPar
U+2500
U+2502
U+250C
U+2510
U+2514
U+2518
U+252C
U+2534
U+253C
&
\sphinxAtStartPar
\sphinxcode{\sphinxupquote{C\sphinxhyphen{}x 8 RET 2500 RET}}
\sphinxcode{\sphinxupquote{C\sphinxhyphen{}x 8 RET 2502 RET}}
\sphinxcode{\sphinxupquote{C\sphinxhyphen{}x 8 RET 250C RET}}
\sphinxcode{\sphinxupquote{C\sphinxhyphen{}x 8 RET 2510 RET}}
\sphinxcode{\sphinxupquote{C\sphinxhyphen{}x 8 RET 2514 RET}}
\sphinxcode{\sphinxupquote{C\sphinxhyphen{}x 8 RET 2518 RET}}
\sphinxcode{\sphinxupquote{C\sphinxhyphen{}x 8 RET 252C RET}}
\sphinxcode{\sphinxupquote{C\sphinxhyphen{}x 8 RET 2534 RET}}
\sphinxcode{\sphinxupquote{C\sphinxhyphen{}x 8 RET 253C RET}}
\\
\sphinxbottomrule
\end{tabulary}
\sphinxtableafterendhook\par
\sphinxattableend\end{savenotes}

\sphinxAtStartPar
Alternatively, run:

\begin{sphinxVerbatim}[commandchars=\\\{\}]
M\PYGZhy{}x insert\PYGZhy{}char RET box draw RET
\end{sphinxVerbatim}

\sphinxAtStartPar
and select from the interactive menu.


\subsection{Quick editing tips}
\label{\detokenize{rst_cheat_sheet:quick-editing-tips}}\begin{itemize}
\item {} 
\sphinxAtStartPar
Use \sphinxstylestrong{spaces} (not tabs) for alignment.

\item {} 
\sphinxAtStartPar
Ensure you are in a \sphinxstylestrong{monospace} buffer. In Doom Emacs, you can toggle variable
pitch fonts off with \sphinxcode{\sphinxupquote{SPC t v}}.

\item {} 
\sphinxAtStartPar
Copy and reuse template boxes instead of redrawing them each time.

\end{itemize}


\subsection{Template box (copy and fill in)}
\label{\detokenize{rst_cheat_sheet:template-box-copy-and-fill-in}}
\begin{sphinxVerbatim}[commandchars=\\\{\}]
┌────────────────────────────┐
│   Example process step     │
└────────────────────────────┘
\end{sphinxVerbatim}

\sphinxAtStartPar
This makes it easy to maintain consistent visual flow diagrams directly inside
your \sphinxcode{\sphinxupquote{.rst}} files without any external graphics tools.


\subsection{Reusable Box Templates (Yasnippet or Tempo)}
\label{\detokenize{rst_cheat_sheet:reusable-box-templates-yasnippet-or-tempo}}
\index{Emacs@\spxentry{Emacs}!Yasnippet@\spxentry{Yasnippet}}\index{Doom Emacs@\spxentry{Doom Emacs}!snippets@\spxentry{snippets}}\index{templates@\spxentry{templates}!box\sphinxhyphen{}drawing@\spxentry{box\sphinxhyphen{}drawing}}\ignorespaces 
\sphinxAtStartPar
If you frequently add diagrams, you can automate box creation using
\sphinxstylestrong{Yasnippet} (bundled with Doom Emacs) or the built\sphinxhyphen{}in \sphinxstylestrong{tempo} snippet system.


\subsection{Yasnippet setup}
\label{\detokenize{rst_cheat_sheet:yasnippet-setup}}
\sphinxAtStartPar
Yasnippet allows you to expand a keyword into a pre\sphinxhyphen{}defined text pattern.
\begin{enumerate}
\sphinxsetlistlabels{\arabic}{enumi}{enumii}{}{.}%
\item {} 
\sphinxAtStartPar
Open your snippets folder (default for Doom):

\begin{sphinxVerbatim}[commandchars=\\\{\}]
\PYGZti{}/.doom.d/snippets/text\PYGZhy{}mode/
\end{sphinxVerbatim}

\sphinxAtStartPar
Create it if it doesn’t exist.

\item {} 
\sphinxAtStartPar
Add a new snippet file, for example \sphinxcode{\sphinxupquote{box}}:

\begin{sphinxVerbatim}[commandchars=\\\{\}]
\PYGZsh{} \PYGZhy{}*\PYGZhy{} mode: snippet \PYGZhy{}*\PYGZhy{}
\PYGZsh{} name: Box diagram template
\PYGZsh{} key: box
\PYGZsh{} \PYGZhy{}\PYGZhy{}
┌────────────────────────────┐
│  \PYGZdl{}\PYGZob{}1:Your process step\PYGZcb{}    │
└────────────────────────────┘
\PYGZdl{}0
\end{sphinxVerbatim}

\item {} 
\sphinxAtStartPar
Reload snippets:

\begin{sphinxVerbatim}[commandchars=\\\{\}]
M\PYGZhy{}x yas\PYGZhy{}reload\PYGZhy{}all
\end{sphinxVerbatim}

\item {} 
\sphinxAtStartPar
Type \sphinxcode{\sphinxupquote{box}} in any buffer and press \sphinxcode{\sphinxupquote{TAB}} → a box appears with an editable field.

\end{enumerate}


\subsection{Tempo setup (alternative)}
\label{\detokenize{rst_cheat_sheet:tempo-setup-alternative}}
\sphinxAtStartPar
If you prefer Emacs’ built\sphinxhyphen{}in \sphinxstylestrong{tempo templates}, add this to your Doom config:

\begin{sphinxVerbatim}[commandchars=\\\{\}]
\PYG{p}{(}\PYG{n+nb}{require}\PYG{+w}{ }\PYG{l+s+ss}{\PYGZsq{}tempo}\PYG{p}{)}

\PYG{p}{(}\PYG{n+nv}{tempo\PYGZhy{}define\PYGZhy{}template}
\PYG{+w}{ }\PYG{l+s}{\PYGZdq{}}\PYG{l+s}{box}\PYG{l+s}{\PYGZdq{}}
\PYG{+w}{ }\PYG{o}{\PYGZsq{}}\PYG{p}{(}\PYG{l+s}{\PYGZdq{}}\PYG{l+s}{┌────────────────────────────┐}\PYG{l+s}{\PYGZbs{}n}\PYG{l+s}{\PYGZdq{}}
\PYG{+w}{   }\PYG{l+s}{\PYGZdq{}}\PYG{l+s}{│  }\PYG{l+s}{\PYGZdq{}}\PYG{+w}{ }\PYG{n+nv}{p}\PYG{+w}{ }\PYG{l+s}{\PYGZdq{}}\PYG{l+s}{  │}\PYG{l+s}{\PYGZbs{}n}\PYG{l+s}{\PYGZdq{}}
\PYG{+w}{   }\PYG{l+s}{\PYGZdq{}}\PYG{l+s}{└────────────────────────────┘}\PYG{l+s}{\PYGZbs{}n}\PYG{l+s}{\PYGZdq{}}\PYG{p}{)}
\PYG{+w}{ }\PYG{l+s}{\PYGZdq{}}\PYG{l+s}{box}\PYG{l+s}{\PYGZdq{}}
\PYG{+w}{ }\PYG{l+s}{\PYGZdq{}}\PYG{l+s}{Insert a Unicode box diagram.}\PYG{l+s}{\PYGZdq{}}\PYG{p}{)}
\end{sphinxVerbatim}

\sphinxAtStartPar
Then restart Emacs or evaluate the code (\sphinxcode{\sphinxupquote{M\sphinxhyphen{}x eval\sphinxhyphen{}buffer}}).
Now you can type:

\begin{sphinxVerbatim}[commandchars=\\\{\}]
M\PYGZhy{}x tempo\PYGZhy{}template\PYGZhy{}box
\end{sphinxVerbatim}

\sphinxAtStartPar
and insert a ready\sphinxhyphen{}made box anywhere.


\subsection{Customization tips}
\label{\detokenize{rst_cheat_sheet:customization-tips}}\begin{itemize}
\item {} 
\sphinxAtStartPar
You can create variants (e.g., \sphinxcode{\sphinxupquote{flowbox}} or \sphinxcode{\sphinxupquote{doublebox}}) with wider lines.

\item {} 
\sphinxAtStartPar
Add these snippets to version control in your Doom config for easy sharing.

\item {} 
\sphinxAtStartPar
Yasnippet placeholders like \sphinxcode{\sphinxupquote{\$\{1:Text\}}} support tab\sphinxhyphen{}cycling between editable fields.

\item {} 
\sphinxAtStartPar
All snippets support UTF\sphinxhyphen{}8 box characters directly—no special input method required.

\end{itemize}

\sphinxAtStartPar
These small helpers make it easy to document process flows or architecture diagrams
right from within Emacs while keeping the style consistent across \sphinxtitleref{.rst} files.


\section{Notes, Warnings, and Tips}
\label{\detokenize{rst_cheat_sheet:notes-warnings-and-tips}}
\index{admonitions@\spxentry{admonitions}}\ignorespaces 
\sphinxAtStartPar
Admonitions help highlight important notes:

\begin{sphinxVerbatim}[commandchars=\\\{\}]
\PYG{p}{..} \PYG{o+ow}{note}\PYG{p}{::}
   This is a note.

\PYG{p}{..} \PYG{o+ow}{warning}\PYG{p}{::}
   This is a warning.

\PYG{p}{..} \PYG{o+ow}{tip}\PYG{p}{::}
   This is a tip.
\end{sphinxVerbatim}

\sphinxAtStartPar
They render with icons and borders in HTML output.


\section{Images}
\label{\detokenize{rst_cheat_sheet:images}}
\index{images@\spxentry{images}!inserting@\spxentry{inserting}}\index{figures@\spxentry{figures}}\ignorespaces 
\sphinxAtStartPar
Images are inserted with the \sphinxcode{\sphinxupquote{.. image::}} or \sphinxcode{\sphinxupquote{.. figure::}} directives.

\sphinxAtStartPar
\sphinxstylestrong{Basic image:}

\begin{sphinxVerbatim}[commandchars=\\\{\}]
\PYG{p}{..} \PYG{o+ow}{image}\PYG{p}{::} \PYGZus{}static/example.png
   \PYG{n+nc}{:alt:} Example image
   \PYG{n+nc}{:width:} 400px
   \PYG{n+nc}{:align:} center
\end{sphinxVerbatim}

\sphinxAtStartPar
\sphinxstylestrong{With caption (use ``figure``):}

\begin{sphinxVerbatim}[commandchars=\\\{\}]
\PYG{p}{..} \PYG{o+ow}{figure}\PYG{p}{::} \PYGZus{}static/diagram.png
   \PYG{n+nc}{:alt:} System architecture
   \PYG{n+nc}{:width:} 80\PYGZpc{}
   \PYG{n+nc}{:align:} center

   System architecture overview.
\end{sphinxVerbatim}

\sphinxAtStartPar
\sphinxstylestrong{Notes:}
\begin{itemize}
\item {} 
\sphinxAtStartPar
Images are usually stored in the \sphinxcode{\sphinxupquote{\_static}} directory.

\item {} 
\sphinxAtStartPar
Use \sphinxcode{\sphinxupquote{:width:}} in pixels or percent for scaling.

\item {} 
\sphinxAtStartPar
\sphinxcode{\sphinxupquote{:align: center}} or \sphinxcode{\sphinxupquote{:align: right}} controls placement.

\item {} 
\sphinxAtStartPar
Always include \sphinxcode{\sphinxupquote{:alt:}} text for accessibility.

\end{itemize}


\section{Code Blocks}
\label{\detokenize{rst_cheat_sheet:code-blocks}}
\index{code blocks@\spxentry{code blocks}}\index{syntax highlighting@\spxentry{syntax highlighting}}\index{literal blocks@\spxentry{literal blocks}}\ignorespaces 
\sphinxAtStartPar
Code blocks are used to show code or console examples with proper syntax highlighting.

\sphinxAtStartPar
\sphinxstylestrong{Generic code block:}

\begin{sphinxVerbatim}[commandchars=\\\{\}]
\PYG{p}{..} \PYG{o+ow}{code\PYGZhy{}block}\PYG{p}{::} python
   def hello(name):
       print(f\PYGZdq{}Hello, \PYGZob{}name\PYGZcb{}!\PYGZdq{})

\PYG{p}{..} \PYG{o+ow}{code\PYGZhy{}block}\PYG{p}{::} bash
   echo \PYGZdq{}Hello, world!\PYGZdq{}
\end{sphinxVerbatim}

\sphinxAtStartPar
\sphinxstylestrong{Key points:}
\begin{itemize}
\item {} 
\sphinxAtStartPar
The language name after \sphinxcode{\sphinxupquote{code\sphinxhyphen{}block::}} controls highlighting
(e.g. \sphinxcode{\sphinxupquote{python}}, \sphinxcode{\sphinxupquote{bash}}, \sphinxcode{\sphinxupquote{json}}, \sphinxcode{\sphinxupquote{ini}}).

\item {} 
\sphinxAtStartPar
Indentation is \sphinxstylestrong{required} — the content must be indented under the directive.

\item {} 
\sphinxAtStartPar
Inline code uses double backticks: \sphinxcode{\sphinxupquote{print("Hi")}}.

\item {} 
\sphinxAtStartPar
For literal, unformatted text (no highlighting), use \sphinxcode{\sphinxupquote{::}} at the end of a paragraph:

\begin{sphinxVerbatim}[commandchars=\\\{\}]
Example\PYG{l+s+se}{::}

\PYG{l+s}{   }\PYG{l+s}{This text is shown exactly as typed.}
\PYG{l+s}{   Useful for quick terminal snippets or pseudo\PYGZhy{}code.}
\end{sphinxVerbatim}

\end{itemize}

\sphinxAtStartPar
\sphinxstylestrong{Tips:}
\begin{itemize}
\item {} 
\sphinxAtStartPar
Use \sphinxcode{\sphinxupquote{code\sphinxhyphen{}block:: rst}} when demonstrating reStructuredText syntax.

\item {} 
\sphinxAtStartPar
Keep lines under \textasciitilde{}80 characters to avoid rendering issues in narrow layouts.

\item {} 
\sphinxAtStartPar
To highlight specific lines, add \sphinxcode{\sphinxupquote{:emphasize\sphinxhyphen{}lines:}}:

\begin{sphinxVerbatim}[commandchars=\\\{\}]
\PYG{p}{..} \PYG{o+ow}{code\PYGZhy{}block}\PYG{p}{::} python
   \PYG{n+nc}{:emphasize\PYGZhy{}lines:} 2

   def main():
       print(\PYGZdq{}Important line!\PYGZdq{})
\end{sphinxVerbatim}

\end{itemize}


\chapter{Indices and tables}
\label{\detokenize{index:indices-and-tables}}\begin{itemize}
\item {} 
\sphinxAtStartPar
\DUrole{xref}{\DUrole{std}{\DUrole{std-ref}{genindex}}}

\item {} 
\sphinxAtStartPar
\DUrole{xref}{\DUrole{std}{\DUrole{std-ref}{modindex}}}

\item {} 
\sphinxAtStartPar
\DUrole{xref}{\DUrole{std}{\DUrole{std-ref}{search}}}

\end{itemize}



\renewcommand{\indexname}{Index}
\printindex
\end{document}